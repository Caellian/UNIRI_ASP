% !TEX TS-program = pdflatex
% !TEX encoding = UTF-8
% !BIB program = biber

\documentclass[12pt,a4paper]{report}

% generira bookmarkove
\usepackage{bookmark}

% Podrška za hrvatski
\usepackage[croatian]{babel}
\usepackage{csquotes}

% Margine
\usepackage[left=3cm,top=2.5cm,right=2.5cm,bottom=3.5cm]{geometry}

% Simboli i naredbe za matematiku
\usepackage{amsmath, amsthm, amssymb}
\usepackage{mathtools}
\usepackage{subcaption} % side-by-side grafovi

% Podrška za slike
\usepackage{graphicx} % \includegraphics [k=v] interface

% Poveznice
\usepackage{hyperref}

% Blokovi s kodom
\usepackage{minted}
\ifx\write18\relax\else
\usemintedstyle{sas}
\fi

% Fonts:
\usepackage[T1]{fontenc} % Allows specifying fonts
\usepackage{helvet} % Helvetica; phv
\usepackage{palatino} % Palatino; ppl

\gdef \class{Algoritmi i strukture podataka}
\gdef \author{Tin Švagelj}
\gdef \uniprogram{Preddimplski studij informatike}
\gdef \semguide{Izv. prof. dr. sc. Marija Brkić Bakarić}
\gdef \author{Tin Švagelj}
\gdef \date{\today}

% Custom commands and redefinitions
\newcommand{\UseFont}[1]{\fontfamily{#1}\selectfont}

\usepackage{titlesec}
\titleformat{\chapter}[hang]
{\normalfont\Large\bfseries}{\thechapter}{20pt}{\Large}

\newcommand{\SubItem}[1]{
    {\setlength\itemindent{15pt} \item[-] #1}
}

\begin{document}

\thispagestyle{empty}

\begin{titlepage}
	\begin{center}
        {
            \bf
            FAKULTET INFORMATIKE I\\
            DIGITALNIH TEHNOLOGIJA\\
            \vspace{2pt}\uniprogram
        }

		\vspace*{\stretch{0.5}}
        {\LARGE Projektni zadatak iz kolegija}\\
        \vspace{8pt}{\LARGE \class}\\
        \vspace*{\stretch{0.5}}
	\end{center}

    {
        \renewcommand{\arraystretch}{1.5}
        \begin{tabular}{l l}
            {\bf Autor:} & {\bf \author} \\
            {Mentor/ica:} & {\semguide} \\
        \end{tabular}
    }

    \vspace*{\stretch{0.25}}
	\begin{center}
		{U Rijeci, \today}
	\end{center}
\end{titlepage}


\pagenumbering{roman}

\tableofcontents
\newpage

\pagenumbering{arabic}
\setcounter{page}{1}

\chapter{Odabir testnih podataka}

Korišteni studentski kod je 031800\textbf{6048}. Kako bi se odredila
datoteka s podatcima za testiranje je unesena formula
\verb|6048%113+1| u interaktivnoj python ljusci, čime je dobiven broj
\texttt{60}. Dobivena vrijednost spada u brojčani raspon
$[47, 64]$ te je zato odabrana treća poveznica za testne
podatke.

Za odabir algoritma za implementaciju je korištena formula
\verb|6048%7+1|, čiji rezultat je \texttt{1} (tj. Quick Sort).

\chapter{Sadržaj testnih podataka}

Kao testne podatke aplikacije su bili korišteni sljedeći podatci o
gubitcima opreme, broju umrli i ranjenih, te broju ratnih zatvorenika u
ratu 2022. između Rusije i Ukrajine:

\begin{minted}[breaklines]{text}
2022-02-25,2,10,7,80,516,49,4,100.0,60.0,0,2,0,,,,,
2022-02-26,3,27,26,146,706,49,4,130.0,60.0,2,2,0,,,,,
2022-02-27,4,27,26,150,706,50,4,130.0,60.0,2,2,0,,,,,
2022-02-28,5,29,29,150,816,74,21,291.0,60.0,3,2,5,,,,,
2022-03-01,6,29,29,198,846,77,24,305.0,60.0,3,2,7,,,,,
2022-03-02,7,30,31,211,862,85,40,355.0,60.0,3,2,9,,,,,
2022-03-03,8,30,31,217,900,90,42,374.0,60.0,3,2,11,,,,,
2022-03-04,9,33,37,251,939,105,50,404.0,60.0,3,2,18,,,,,
2022-03-05,10,39,40,269,945,105,50,409.0,60.0,3,2,19,,,,,
2022-03-06,11,44,48,285,985,109,50,447.0,60.0,4,2,21,,,,,
2022-03-07,12,46,68,290,999,117,50,454.0,60.0,7,3,23,,,,,
2022-03-08,13,48,80,303,1036,120,56,474.0,60.0,7,3,27,,,,,
2022-03-09,14,49,81,317,1070,120,56,482.0,60.0,7,3,28,,,,,
2022-03-10,15,49,81,335,1105,123,56,526.0,60.0,7,3,29,,,,,
2022-03-11,16,57,83,353,1165,125,58,558.0,60.0,7,3,31,,,,,
2022-03-12,17,58,83,362,1205,135,62,585.0,60.0,7,3,33,,,,,
2022-03-13,18,74,86,374,1226,140,62,600.0,60.0,7,3,34,,,,,
2022-03-14,19,77,90,389,1249,150,64,617.0,60.0,8,3,34,,,,,
2022-03-15,20,81,95,404,1279,150,64,640.0,60.0,9,3,36,,,,,
2022-03-16,21,84,108,430,1375,190,70,819.0,60.0,11,3,43,10.0,,,,
\end{minted}

Prvih 20 redova podataka su bili pročišćeni od dupliciranih vrijednosti
i datuma uporabom jezičnog modela. Zatim pohranjeni u jednom redu
(zarezima odvojeni) u \verb|data.in| datoteku.

\chapter{Ispis programa za sortiranje}

Prilikom paljenja programa, učitava se datoteka s testnim podatcima.

Ostatak toka izvođenja programa je dokumentiran stdout ispisom:

\inputminted[breaklines]{text}{output.txt}

\end{document}
